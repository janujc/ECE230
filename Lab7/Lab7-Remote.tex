\documentclass[12pt]{../manual}
%____________________________________________________________________________
%
%	TITLE AND TABLE OF CONTENTS
%____________________________________________________________________________
\begin{document}
\makeheader{Lab 7}
\begin{center}
\textbf{\huge ECE 230L - LAB 7}\\~\\
\textbf{\large BASIC DIGITAL CIRCUITS}\\~\\
\rule{6.5in}{0.5mm}\\
\end{center}

\tableofcontents

\listoffigures

\newpage
%____________________________________________________________________________
%
%	BODY
%____________________________________________________________________________
\section{Objectives of this Laboratory}
The objectives of this laboratory session are as follows:
\begin{itemize}
\item Determine experimentally the truth tables of Discrete Diode Logic Gates,
\item Determine experimentally the truth tables of Discrete MOS Inverter Circuit,
\item Measure the static and switching characteristics of a discrete MOS Inverter Circuit, and
\item Create and experimentally determine the truth table of a Discrete NAND Gate
\end{itemize}

\section{Discrete Diode Logic OR and AND Gates}
Diodes can be used to significantly reduce circuit cost, complexity, and save precious board space. By implementing discrete-device logic instead of using OR and AND gates, whole integrated circuits (ICs) can be replaced in a circuit. In general, ICs with 50\% or fewer gates being used stand to gain by being replaced by discrete components. Diode logic takes advantage of the diode properties to build useful circuits that can reduce component count and simplify circuits. A few simpler components can be inserted to perform the same function of large integrated circuits more cheaply and more directly. Often, it is more convenient to build a logic gate than use a pre-manufactured chip. When more than a few inputs are desired, building your own logic gates can be the only way to get the desired number. For instance, a 7-input OR gate is easily implemented just by adding more diodes to a basic 2-input gate. You may not always have the ICs that you need available, and ordering them and then waiting for the delivery takes time. You will always have resistors and diodes, and this is another reason for using logic gates made from discrete components.

\newpage
\subsection{Diode OR Gate}
Consider the diode as a simple switch. It is closed (ON) when the voltage on the anode or p-side is higher than that on the cathode or n-side. Current flows in the direction of the arrow in the diode's schematic symbol. The logic symbol of an OR gate and its discrete diode implementation are shown in Figure \ref{fig:OR}. In an OR gate, the output is `1' if either of the inputs are `1'. In other words, if either of the inputs has a high voltage, its diode will conduct and current will flow to the output. A high voltage will appear across the resistor, equal to the input voltage minus a voltage drop across the silicon diode. If both of the inputs are `0', then neither of the diodes will conduct and the gate?s output voltage will be zero. 

\begin{itemize}
\item[$\square$] Wire the diode OR gate using 1N4148 Si diodes on a breadboard and verify its truth table by applying 0 and \SI{5}{\volt} (4 combinations) to the inputs A and B. To apply 0V, wire the input to ground. Measure the value of the output voltage. Note the diode voltage drop and its effect on the output voltage.
\end{itemize}

\begin{figure}[ht!]
\centering
\begin{tabular}{m{5cm} m{5cm}}
\begin{circuitikz}[american]
\draw (0,0) 	node[american or port, name=A] {};
\draw (A.in 1) 	to[short, -o] ++(-0.5,0) node[left] {A};
\draw (A.in 2) 	to[short, -o] ++(-0.5,0) node[left] {B};
\draw (A.out) 	to[short, -o] ++(0.5,0) node[right] {C};
\end{circuitikz} &
\begin{circuitikz}[american]
\draw (0,0) 	node[left] {A}
				to[D*, o-*] ++(3,0) 
				to[short, -o] ++(2,0) node[right] {C};
\draw (3,0) 	to[short] ++(0,-1)
				to[R, l=$R$, a=\SI{10}{\kilo\ohm}] ++(0,-2) node[ground, below] {};
\draw (0,-1)	node[left] {B}
				to[D*, o-*] ++(3,0);
\end{circuitikz}
\end{tabular}
\caption{OR Gate Symbol and Discrete Diode Implementation}
\label{fig:OR}
\end{figure}

\begin{itemize}
\item[$\square$] Repeat the exercise above with a Ge Diode and note the diode voltage drop and its effect on the output table. Which circuit is "better"?
\end{itemize}

Note that `$+$' is used to represent the OR operation. Hence, in Figure \ref{fig:OR} we have 
\begin{align}
C = A + B.
\end{align}

\newpage
\subsection{Diode AND Gate}
Consider the diode AND gate shown in Figure \ref{fig:AND}. Its circuit is similar to the OR-gate circuit except that the diode connections (anodes and cathodes) are switched, and the resistor is connected to a power supply of \SI{5.0}{\volt}, instead of ground. The output of an AND gate is `1' only if both inputs are `1'. In the diode implementation, if either input is `0', then the diode will conduct and the output voltage will be effectively shorted (through the diode) to ground. If both inputs are `1' , then neither of the diodes will be conducting and the output voltage will be \SI{5.0}{\volt} (logical `1') because nearly all the voltage drops across the \SI{10}{\mega\ohm} resistor. This operation yields the desired result. Note that again, due to the voltage drop across a conducting silicon PN-junction diode, the actual `low' output voltage is higher than the `low' voltage of \SI{0}{\volt} applied at the input of the gate.

\begin{itemize}
\item[$\square$] Wire the diode AND gate using 1N4148 Si diodes on a breadboard and verify its truth table by applying 0 and \SI{5}{\volt} (4 combinations) to the inputs A and B. Measure the output voltage. Note the diode voltage drop and its effect on the output voltage.
\end{itemize}

\begin{figure}[ht!]
\centering
\begin{tabular}{m{5cm} m{5cm}}
\begin{circuitikz}[american]
\draw (0,0) 	node[american and port, name=A] {};
\draw (A.in 1) 	to[short, -o] ++(-0.5,0) node[left] {A};
\draw (A.in 2) 	to[short, -o] ++(-0.5,0) node[left] {B};
\draw (A.out) 	to[short, -o] ++(0.5,0) node[right] {C};
\end{circuitikz} &
\begin{circuitikz}[american]
\draw (0,0) 	node[left] {A}
				to[D*, invert, o-*] ++(3,0) 
				to[short, -o] ++(2,0) node[right] {C};
\draw (3,0) 	to[R, a=$R_1$, l=\SI{10}{\kilo\ohm}] ++(0,2) node[vdd, above] {$V_{\mathrm{DD}}$};
\draw (3,0) 	to[short] ++(0,-1)
				to[R, l=$R_2$, a=\SI{10}{\mega\ohm}] ++(0,-2) node[ground, below] {};
\draw (0,-1)	node[left] {B}
				to[D*, invert, o-*] ++(3,0);
\end{circuitikz}
\end{tabular}
\caption{AND Gate Symbol and Discrete Diode Implementation}
\label{fig:AND}
\end{figure}

Note that `$\cdot$' is used to represent the AND operation. Hence, in Figure \ref{fig:AND} we have 
\begin{align}
C = A \cdot B.
\end{align}

\newpage
\section{Discrete MOS Inverter Circuit}
An inverting function cannot be implemented with diodes and resistors alone. A transistor is needed to provide the inverting action. In this lab, you will use the BS170 N-type Metal Oxide Semiconductor Field Effect Transistor (NMOSFET). If the voltage present at the gate of the transistor is above \SI{0.7}{\volt} (high), the transistor will conduct, reducing the output voltage to logical `0'. If the input voltage is logical `0' (low), then the transistor does not conduct, and zero voltage drops across the load resistor which results in the output voltage being `1'. A current-limiting resistor at the gate is always needed, otherwise excessive gate current might destroy the transistor. The circuit used in the electrical characterization of the MOS inverter with a resistive load is shown in Figure \ref{fig:MOS}. Digital integrated circuits are circuits based on the principles of Boolean algebra. The binary logic variables in this algebra can assume only one of two possible values which are called the logical `0' and `1' levels. In the electronic circuit implementation of Boolean algebra, these variables correspond to the values designated by the voltages $V_\mathrm{OL}$ and $V_\mathrm{OH}$.

\begin{figure}[ht!]
\centering
\begin{tabular}{m{5cm} m{5cm}}
\begin{circuitikz}[american]
\draw (0,0) 	node[american not port, name=A] {};
\draw (A.in 1) 	to[short, -o] ++(-0.5,0) node[left] {A};
\draw (A.out) 	to[short, -o] ++(0.5,0) node[right] {B};
\end{circuitikz} &
\begin{circuitikz}[american]
\draw (0,0)		node[nigfete, solderdot, name=N]{};
\ctikzset{diodes/scale=0.3}
\draw ($(N.S) + (0,0.2)$) to[short,*-] ++(0.5,0)
			to[zzD*] ++(0,1.15)
			to[short, -*] ++(-0.5,0);
\draw (N) circle [radius=25pt];
\draw ($(N) + (0.9,0)$) node[right] {BS170};
\draw (N.S)	to[short] ++(0,-1) node[ground,  below] {};
\draw (N.D) to[short, -*] ++(0,0.5) node[anchor=east, name=A] {}
			to[R, l=$R_L$, a=\SI{240}{\kilo\ohm}] ++(0,2) node[vdd] {$V_\mathrm{DD}$};
\draw (A) 	to[short, -o] ++(1.5,0) node[right] {B};
\draw (N.G) to[short, -o] ++(-1,0) node[left] {A};
\end{circuitikz}
\end{tabular}
\caption{Inverter Gate Symbol and Discrete MOS Implementation}
\label{fig:MOS}
\end{figure}

We usually denote negation with a bar above a character. Hence, in Figure \ref{fig:MOS} we have 
\begin{align}
B = \overline{A}
\end{align}

The static voltage and switching characteristics of an inverter can be characterized and serve as a measure of the inverter's overall performance. The voltages $V_\mathrm{OL}$ and $V_\mathrm{OH}$ are always between 0 and $V_\mathrm{DD}$. The static voltage characteristics of an inverter is shown in Figure \ref{fig:static}. The switching characteristics of an inverter is shown in Figure \ref{fig:switch}.

\begin{enumerate}
\item Wire the diode Inverter gate using a BS170 MOSFET on a breadboard as shown in Figure \ref{fig:MOS} and verify its truth table by applying \SI{0}{\volt} and \SI{5}{\volt} to the input A. Construct $R_{L}$ using a combination of resistors in parallel and in series. You should be able to come within \SI{10}{\ohm} of $R_L$ with 4 resistors. Measure the value of the output voltage. Note the voltage drops across the inverter circuit and their effect on the output voltage.
\item Using the M1K, measure the static voltage-transfer characteristics (SVTC)
$V_\mathrm{OUT}(VIN)$ for this inverter implemented with the NMOSFET BS170 with the same configurations as above. Think about where you need to connect Ch. A and Ch. B from the M1K to the circuit. Set Ch. A to voltage source, triangle wave, $V_\mathrm{PP}$, and \SI{10}{\hertz}. Connect Ch. A GND and Ch. B GND to the GND of the Power Source GND. Save a screenshot of the graphs on Pixelpulse 2 and determine $V_\mathrm{OL}$ and $V_\mathrm{OH}$.
\item Measure the output-voltage waveform $V_\mathrm{OUT}(t)$ for a square-wave input-voltage waveform with $V_\mathrm{PP}$ = \SI{5}{\volt} and f = \SI{1}{\kilo\hertz} with the M1K. Observe the input and output waveforms on the oscilloscope simultaneously. Save a screenshot of the graphs on the Pixelpulse 2. Determine the high-to-low transition time (tp.HL) and the low-to-high transition time (tp.LH). Turn up the input frequency of the square wave keeping the maximum input voltage equal to $V_\mathrm{PP}$ = \SI{5}{\volt}. At what frequency does the output waveform begin to degrade? Be as quantitative
as possible. Save a screenshot of the graphs.
\end{enumerate}

\begin{figure}[ht!]
\centering
\includegraphics[width=0.7\textwidth]{./figures/staticCharacteristicsInverter.png}
\caption{Static Voltage Characteristics of Inverter Circuit}
\label{fig:static}
\end{figure}

\begin{figure}[ht!]
\centering
\includegraphics[width=0.6\textwidth]{./figures/switchCharInv.png}
\caption{Switching Characteristics of Inverter Circuit}
\label{fig:switch}
\end{figure}


\newpage
\section{Exploration}
In this exploration, you will use what you have learned about discrete logic circuits to complete the following:
\begin{enumerate}
\item Build a NAND gate from the combination of discrete AND gates using diodes and Inverter gates using NMOSFET transistors.
\item One very useful application of an Inverter gate is the Inverter Ring Oscillator shown in Figure \ref{fig:ring}. You will not need to build a ring oscillator but will need an understanding of this application.
\item Another very useful implementation of NAND gates is the Debounced Switch or SR Latch. A Debounce Switch prevents pushbutton switching events from triggering more than On/Off occurrence in a circuit. Deboucing inputs is critical in many logic circuit applications. A Debounce Swtich or SR Latch can be completely constructed using discrete NAND gate logic when wired as shown in Figure \ref{fig:SR}. You will not need to build a Debounced Switch or SR Latch but will need an understanding of these applications.
\end{enumerate}

\subsection{Discrete NAND Gate}
Using a combination of the AND gates using diodes and Inverter gates using NMOSFET transistors built above, build a discrete NAND gate. Draw a circuit diagram of the circuit you designed and verify the NAND gate truth table. 

\begin{figure}[ht!]
\centering
\begin{circuitikz}[american]
\draw (0,0)		node[american nand port, name=A] {};
\draw (A.in 1) 	to[short, -o] ++(-0.5,0) node[left] {A};
\draw (A.in 2) 	to[short, -o] ++(-0.5,0) node[left] {B};
\draw (A.out) 	to[short, -o] ++(0.5,0) node[right] {C};
\end{circuitikz}
\caption{NAND Gate Symbol}
\label{fig:NAND}
\end{figure}

\newpage
\subsection{Ring Oscillator}
The propagation delay of a switching circuit is a measure of the maximum speed or minimum time
for an input change to pass to the device?s output. Measurements in this mode are made by setting
up an odd number of inverters or inverting gates in a ring, that is, with the output of one inverter
connected to the input of the next inverter. In this mode, a logic steady-state cannot be reached.
The results is a passing along of the logic mismatch?a form of oscillation.
To measure the propagation delay of the discrete inverter built above, arrange an odd number of
inverters (at least 3) in series. No input excitation at the first inverter is needed; the circuit should
oscillate when the inverters are on (i.e. when VDD is applied to each discrete MOSFET inverter).
The output of the Ring Oscillator should be a square wave of amplitude 0 to VDD.

\begin{figure}[ht!]
\centering
\begin{circuitikz}[american]
\draw (0,0) 	node[american not port, name=A] {};
\draw (2,0) 	node[american not port, name=B] {};
\draw (4,0) 	node[american not port, name=C] {};
\draw (A.out)	-- (B.in);
\draw (B.out) 	-- (C.in);
\draw (C.out) 	to[short, -o] ++(2,0);
\draw ($(C.out) + (1,0)$) 	to[short, *-] ++(0,-2) node[name=D] {}
				to[short] ($(D -| A.in) + (-1,0)$)
				to[short] ++(0,2)
				to[short] (A.in);
\end{circuitikz}
\caption{Ring Oscillator Using 3 Inverters}
\label{fig:ring}
\end{figure}

\subsection{Debounced Switch}
In many applications, especially in digital circuits, you need a manual switch to
set the logic state at some point in the circuit. The issue is that switch can often
``bounce'' upon actuation. A bouncy switch is one that produces multiple
switching operations in a single button press. This leads to a rapid series of
alternating logic states for a few milliseconds while the switch contact is
stabilizing. The bounce is due to inductive ringing rather than mechanical
toggling. We use an SR latch to debounce the switch in hardware by latching
the state of the switch as a digital signal. Once the ringing stops, we can safely
assume that the signal has been latched successfully.

\begin{figure}[ht!]
\centering
\begin{circuitikz}[american]
\draw (0,0)		node[cute spdt down arrow, rotate=90, name=A] {};
\draw (A.in)	node[ground, below] {};
\draw (A.out 1)	to[short, -*] ++(-1,0) node[name=B] {}
				to[R, l=$R_1$, a=\SI{1}{\kilo\ohm}] ++(0,3) node[vdd] {$V_\mathrm{DD}$};
\draw (B)		to[short] (B |- A.in) node[ground, below] {};
\draw (A.out 2)	to[short, -o] ++(3,0) node[right] {Out};
\draw ($(A.out 2) + (1,0)$) to[R, l=$R_2$, a=\SI{1}{\kilo\ohm}, *-] ++(0,3) node[vdd] {$V_\mathrm{DD}$};
\end{circuitikz}
\caption{Simple Double-Pole Single-Throw (DPST) Switch}
\label{fig:DPST}
\end{figure}

\newpage
The issue of switch bounce can be corrected through the use of a pair of NAND
gates.

\begin{figure}[ht!]
\centering
\begin{circuitikz}[american]
\draw (0,0)		node[cute spdt down arrow, rotate=90, name=A] {};
\draw (4,-3)	node[american nand port, name=C] {};
\draw (4,0.074)	node[american nand port, name=D] {};
\draw (A.in)	node[ground, below] {};
\draw (A.out 1)	to[short, -*] ++(-1,0) node[name=B] {}
				to[R, l=$R_1$, a=\SI{1}{\kilo\ohm}] ++(0,3) node[vdd] {$V_\mathrm{DD}$};
\draw (B)		to[short] (B |- C.in 2)
				to[short] (C.in 2);
\draw (A.out 2)	to[short] (D.in 1);
\draw ($(A.out 2) + (1,0)$) to[R, l=$R_2$, a=\SI{1}{\kilo\ohm}, *-] ++(0,3) node[vdd] {$V_\mathrm{DD}$};
\draw (C.out)	to[short] ++(0.5,0)
				to[short] ++(0,1)
				to[short] ++(-3,1) |- (D.in 2);	
\draw (D.out)	to[short, -o] ++(2,0) node[right] {Out};
\draw ($(D.out) + (0.5,0)$) to[short, *-] ++(0,-1)
				to[short] ++(-3,-1) |- (C.in 1);
\end{circuitikz}
\caption{Debounced Switch Using 2 NAND Gates (SR Latch)}
\label{fig:modDPST}
\end{figure}

\newpage
\section{IC CMOS NOT, NOR, and NAND Gates}
The discrete component logic gates built in this laboratory form the building blocks of all semiconductor logic circuits built and used in industry. Because logic functions are so common, specific Integrated Circuits (ICs) have been built to the highest tolerances to perform on/off functions which are used in electronic devices. There exist common equivalents to the OR, AND, and NOT gates built in this laboratory in individual commercially available IC packages. It is most common to find the NOT versions of the OR and AND functions as ICs (in fact, the OR and AND functions themselves are usually created using a NOT plus OR gate or a NOT plus AND gate in an IC package). The common equivalent devices available commercially are as follows:
\begin{itemize}
\item CD4069 CMOS Inverter
\item CD4001 CMOS NOR gate
\item CD4011 CMOS NAND gate
\end{itemize}

As you can imagine, because these ICs are specifically built for the purpose they serve, their static voltage-transfer characteristics, static noise margins, static power-supply-current characteristics, and static power consumption are much better than the discrete version equivalents. In addition, their transient switching characteristics (measured by observing their response to a square-wave input-voltage waveform) are much faster than the discrete equivalents. For this reason, and due to the low cost of such integrated circuit devices, they are usually used in application specific integrated circuit (ASIC) designs.

The latch is the most basic memory element in digital electronics, but it is seldom used by itself. Instead, digital designers generally prefer flip-flops (clocked latches) as primitive memory elements because they are less susceptible to glitches. Below is the schematic of an SR flip-flop, which is an SR latch with a gated clock input. The clock signal synchronizes any changes to occur only when the clock is high (An alternative is to use a D flip-flop, which is synchronized to a near-instantaneous clock edge.) The main point is that reducing the duration in which the memory element is transparent to changes reduces the likelihood of a glitch being incorrectly recorded into it.

\begin{figure}[ht!]
\centering
\begin{circuitikz}[american]
\draw (0,0)		node[american and port, name=A] {};
\draw (0,3.5)	node[american and port, name=B] {};
\draw (3,0.28)	node[american nor port, name=C] {};
\draw (3,3.22)	node[american nor port, name=D] {};
\draw (A.in 2)	to[short, -o] ++(-2,0) node[left] {$S$};
\draw (B.in 1)	to[short, -o] ++(-2,0) node[left] {$R$};
\draw (A.in 1)	to[short] ++(-1,0) node[name=E] {}
				to[short] (E |- B.in 2) node[name=F] {}
				to[short] (B.in 2);
\draw ($(E)!0.5!(F)$) to[short, *-o] ++(-1,0) node[left] {$E$};
\draw (A.out) |- (C.in 2);
\draw (B.out) |- (D.in 1);
\draw (C.out)	to[short, -o] ++(2,0) node[right] {$\overline{Q}$};
\draw (D.out)	to[short, -o] ++(2,0) node[right] {$Q$};
\draw ($(D.out) + (0.5,0)$) to[short, *-] ++(0,-1)
				to[short] ++(-3,-1) |- (C.in 1);
\draw ($(C.out) + (0.5,0)$) to[short, *-] ++(0,1)
				to[short] ++(-3,1) |- (D.in 2);	
\end{circuitikz}
\caption{Gate-Level Diagram of a Clocked NAND-Gate SR Flip-flop}
\label{fig:SR}
\end{figure}

\newpage
\section{Questions}
\begin{enumerate}
\item Explain how a ring oscillator works to create a signal without the need for any input
\item If the period for a three-stage ring oscillator is 17.4 ms, what is the propagation/transition time for one gate? (HINT: gate = stage) 
\item What signal frequency would the oscillator in Question 2 be producing?
\item What is switch bounce and how does it occur?
\item How does a SR latch work to debounce a switch?
\end{enumerate}


%____________________________________________________________________________
%
%	Grading Rubric
%____________________________________________________________________________
\newpage
\def\arraystretch{1.2}
\phantomsection
\addcontentsline{toc}{section}{Grading Rubric}
\markboth{Grading Rubric}{Grading Rubric}
\hspace{0pt}
\vfill % used to center table vertically on page
\begin{table}[ht!]
\caption{ECE 230L Laboratory 7 Grading Rubric}
\centering
\begin{tabular}{l|c} \hline
Criteria & Points Possible \\ \hline \hline
\textbf{Diode OR Gate}			& \textbf{16} \\
Circuit Diagram 				& 3 \\
Truth Table Verified			& 3 \\
Variable diode drop values given for Si Diode & 3 \\
Variable diode drop values given for Ge Diode & 3 \\
Justification of whether circuit is ``better'' & 4 \\ \hline
\textbf{Diode AND Gate}			& \textbf{9} \\
Circuit Diagram 				& 3 \\
Truth Table Verified			& 3 \\
Diode Drop Values Noted			& 3 \\ \hline
\textbf{Discrete MOS Inverter Circuit}		& \textbf{44} \\
Circuit Diagram 				& 3 \\
Truth Table Verified			& 3 \\
Voltage Lost Across Circuit 	& 3 \\
Image of Pixelpulse 2 when triangle wave is applied & 5 \\
$V_\mathrm{OL}$ from Pixelpulse 2 graph 	& 3 \\
$V_\mathrm{OH}$ from Pixelpulse 2 graph	& 3 \\
Image of Pixelpulse 2 when square wave is applied & 5 \\
High-to-Low Transition Time $(t_\mathrm{p \cdot HL})$ & 5 \\
Low-to-high transition time $(t_\mathrm{p \cdot LH})$ & 5 \\
Degraded image of Pixelpulse 2 when square wave is applied & 5 \\
Degradation frequency with explanation & 4 \\ \hline
{\bf Exploration: Discrete NAND Gate w/ Applications} & {\bf 6} \\
Discrete NAND gate circuit diagram & 3 \\
NAND Gate Truth Table Verified & 3 \\ \hline
{\bf Questions} & {\bf 25} \\ 
Question 1 & 5 \\
Question 2 & 5 \\
Question 3 & 5 \\
Question 4 & 5 \\
Question 5 & 5 \\ \hline \hline
{\bf Total}						& {\bf 100} \\ \hline
\end{tabular}
\end{table}
\vfill % used to center table vertically on page
\end{document}
