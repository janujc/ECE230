\documentclass[12pt]{../manual}
%____________________________________________________________________________
%
%	TITLE AND TABLE OF CONTENTS
%____________________________________________________________________________
\begin{document}
\makeheader{Lab 7}{Fall 2019} % Change Semester and Year as needed
\begin{center}
\textbf{\huge ECE 230L - LAB 7}\\~\\
\textbf{\large DEVICE NON-IDEALITIES}\\~\\
\rule{6.5in}{0.5mm}\\
\end{center}

\tableofcontents

\listoffigures

\newpage
%____________________________________________________________________________
%
%	BODY
%____________________________________________________________________________
\section{Objectives of this Laboratory}
The objectives of this laboratory session are as follows:
\begin{itemize}
\item To gain understanding of some of the less than ideal behavior of devices and circuits explored in previous laboratories. 
\item To explore methods for measuring these non-idealities in a less structured lab environment using the tools presented during the course of the semester.
\item To work with a group in the course to explore these methods.
\item To present group findings to fellow students in the course in a lab presentation.
\end{itemize}

\section{Experimental Exploration Format}
\begin{itemize}
\item This lab will be conducted in groups of two or three. 
\item Each group will be assigned an exploration
\item Complete the exploration and form a brief presentation to share with the lab section---\textit{you have one hour}
\item Each group will present their findings to the entire lab
\end{itemize}

\newpage
\section{Experimental Explorations}
There are five possible explorations:
\subsection{Thermal Effects on PN-Junction Diode \& MOSFET}
\textit{To be completed in a group of 2 or 3.}
\subsubsection*{PN-Junction Diode}
\begin{enumerate}
\item Construct the following circuit on a breadboard:
\end{enumerate}

\def\coord(#1){coordinate(#1)}
\def\coord(#1){node[circle, red, draw, inner sep=1pt,pin={[red, overlay, inner sep=0.5pt, font=\tiny, pin distance=0.1cm, pin edge={red, overlay,}]45:#1}](#1){}}

\begin{myfigure}[label=fig:PNTest]{PN-Junction Diode Test Circuit}{PN-Junction Diode Test Circuit}
\centering
\begin{circuitikz}[scale=2]
\ctikzset{resistors/scale=1.5,batteries/scale=1.5,instruments/scale=1.5,diodes/scale=1.5}
\draw
(0,2) 	to[battery, l^=$V_{DC}$, i=$i_D$] ++(0,-2)
(0,2)	to[R, l_=$R$]		++(2,0)
		to[rmeter, t=A] ++(2,0)
		to[D*, l_=$D$]		++(0,-2)
		to[short]	++(-4,0)
;\end{circuitikz}
\end{myfigure}

\begin{enumerate}
\setcounter{enumi}{1}
\item Run the {\tt singleloop.vi} script from 0 to 6 V with 100 steps. This produces $I_D(V_D)$.
\item Repeat the above, but with a voltmeter over the diode, to measure $V_{PN}$. Combine the results to produce the graph $I_D(V_{PN})$.
\item Now, obtain thermal paste from your TA and apply it to the diode. Obtain a soldering iron and heat it to its lowest setting. Apply the soldering iron to the diode to allow it to heat it.
\item Repeat steps (1) -- (3), and compare the results.
\end{enumerate}

\newpage
\subsubsection*{MOSFET}
\begin{myfigure}[label=fig:MOSTest]{Circuit used to characterize an NMOSFET}{Circuit used to characterize an NMOSFET}
\centering
\begin{circuitikz}[american]
\draw (0,0) 	node[nigfete, solderdot](Q1) {};
\draw (Q1.S) 	to[short, -*] (0,-3) node[ground](GND) {};
\draw (Q1.D) 	to[short] ++(0,2)
				to[R, l=$R_{DS}$, i<=$I_{DS}$, a=\SI{100}{\ohm}] ++(4,0)
				to[battery, l=$V_{\mathrm{DS~Source}}$] (4,-3)
				to[short] ++(-4,0);
\draw (Q1.G)	to[short] ++(-3,0) coordinate(VGS)
				to[battery, l_=$V_{\mathrm{GS~Source}}$, i=$I_{GS}$] (VGS |- GND)
				to[short] (0,-3);
\draw (0.5,2.1)	node[] {$+$};
\draw (Q1.east)	node[right=2mm]{$V_{DS}$};
\draw (0.5,-2.5)node[] {$-$};
\draw (-0.5,-0.7)	node[] {$+$};
\draw (-0.6, -1.6)	node[] {$V_{GS}$};
\draw (-0.5,-2.5)	node[] {$-$};
\end{circuitikz}
\end{myfigure}

\newpage
\subsection{MOSFET Amplifier Gain and Load Limits}
\begin{figure}[ht!]
\centering
\begin{circuitikz}
\draw
(0,0) node[nmos, arrowmos](nmos1){} 
(nmos1.B) 	to[short] ++(-2,0) coordinate(p1) 
			to[R, a=$R_2$, l=\SI{820}{\kilo\ohm}, *-*] ++(0,-3)  
(p1) 		to[C, a=$C_1$, l=\SI{10}{\micro\farad}] ++(-3,0)
			to[battery, a=$v_i(t)$] ++(0,-3) node(gndl){}
(p1) 		to[R, l=$R_1$, a=\SI{1}{\mega\ohm}, *-] ++(0,3) node(pwrl){}
(nmos1.E) 	to[R, a=$R_4$, l=\SI{270}{\ohm}, *-*] (0,-3)
(nmos1.E) 	to[short] ++(3,0) 
			to[C, a=$C_2$, l=\SI{10}{\micro\farad}] (3,-3) node(gndr){}
			to[short] (gndl)
(nmos1.C) 	to[short, -o] ++(3,0) node[right]{$v_o(t)$}
(nmos1.C) 	to[R, l=$R_3$, a=\SI{680}{\ohm}] (nmos1.C |- pwrl) node(pwrr){}
			to[short] (pwrl)
($(pwrl)!0.5!(pwrr)$) to[short, *-o] ++(0,1) node[above=2mm] {\SI{15}{\volt}}
($(gndl)!0.5!(gndr)$) node[ground] {}
;\end{circuitikz}
\end{figure}

\begin{figure}[ht!]
\centering
\begin{circuitikz}
\draw
(0,0) node[nmos, arrowmos](nmos1){} 
(nmos1.B) 	to[short] ++(-2,0) coordinate(p1) 
			to[R, a=$R_2$, l=\SI{820}{\kilo\ohm}, *-*] ++(0,-3)  
(p1) 		to[C, a=$C_1$, l=\SI{10}{\micro\farad}] ++(-3,0)
			to[battery, a=$v_i(t)$] ++(0,-3) node(gndl){}
(p1) 		to[R, l=$R_1$, a=\SI{1}{\mega\ohm}, *-] ++(0,3) node(pwrl){}
(nmos1.E) 	to[R, a=$R_4$, l=\SI{270}{\ohm}, *-*] (0,-3)
(nmos1.E) 	to[short] ++(3,0) 
			to[C, a=$C_2$, l=\SI{10}{\micro\farad}, -*] (3,-3) node(gndr){}
(nmos1.C) 	to[short, -*] ++(6,0) node(vout) {}
			to[C, a=$C_L$] ++ (0,-1.5)
			to[R, a=$R_L$] (6,-3)
			to[short] (gndl)
(vout)		to[short, -o] ++(1,0) node[right]{$v_o(t)$}
(nmos1.C) 	to[R, l=$R_3$, a=\SI{680}{\ohm}] (nmos1.C |- pwrl) node(pwrr){}
			to[short] (pwrl)
($(pwrl)!0.5!(pwrr)$) to[short, *-o] ++(0,1) node[above=2mm] {\SI{15}{\volt}}
($(gndl)!0.5!(gndr)$) node[ground] {}
;\end{circuitikz}
\end{figure}

\subsection{MOSFET Input and Output Resistance}

\subsection{MOSFET Inverter maximum clock frequency with external capacitive load}

\subsection{Zener diode (Reverse breakdown)}

\begin{figure}[ht!]
\centering
\begin{circuitikz}[scale=2]
\ctikzset{resistors/scale=1.5,batteries/scale=1.5,instruments/scale=1.5,diodes/scale=1.5}
\draw
(0,2) 	to[battery, l^=$V$, i=$i_D$] ++(0,-2)
		to[short] ++(2,0) node(bot) {}
		to[zzDo, a=$D$, *-*] ++(0,2) 
(0,2)	to[R, l_=$R$] ++(2,0) node(top) {}
		to[short, -o] ++(1.5,0) node[below=2mm] {$+$}
(bot)	to[short, -o] ++(1.5,0) node[above=2mm] {$-$}
($(top)!0.5!(bot)$) node[right=20mm] {$V_{\mathrm{out}} = V_z$}
;\end{circuitikz}
\end{figure}
%____________________________________________________________________________
%
%	GRADING RUBRIC
%____________________________________________________________________________
\newpage
\phantomsection % needed in order to link from table of contents to here
\section*{Grading Rubric}
\addcontentsline{toc}{section}{Grading Rubric} % adds section*{} to table of contents
\markboth{Grading Rubric}{Grading Rubric}
\end{document}
