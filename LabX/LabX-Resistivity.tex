\documentclass[12pt]{../manual}
%____________________________________________________________________________
%
%	TITLE AND TABLE OF CONTENTS
%____________________________________________________________________________
\begin{document}
\makeheader{Lab X}{Fall 2019}
\begin{center}
\textbf{\huge ECE 230L - LAB X}\\~\\
\textbf{\large SILICON WAFER CHARACTERIZATION \\
RESISTIVITY, CARRIER CONCENTRATION, MOBILITY}\\~\\
\rule{6.5in}{0.5mm}\\
\end{center}

\textit{Note: Prior to completing this laboratory, a PreLab should be completed before the FIRST lab section occurring the week of your scheduled lab.  This PreLab consists of entering your average measured 4-point probe sheet resistance, sample thickness, and resistivity values for each of the 3 samples probed in the SMIF.  To complete this laboratory, the entire class data set will be used.  Type in your data at the link provided on the ECE 230L Sakai site by the due date, above using the online data entry link provided on Sakai.}

\tableofcontents

\newpage
%____________________________________________________________________________
%
%	Begin{PROBLEMS}
%____________________________________________________________________________
\phantomsection
\section*{PreLab: Calculate Resistivity from Van der Pauw Probe Measurement}
\addcontentsline{toc}{section}{PreLab}
\markboth{PreLab}{PreLab}

The goal of this portion of the laboratory is to determine, from the sheet resistance data measured for the Silicon wafers in the SMIF, the resistivity of each of the wafers.

The four-point probe Van der Pauw method that was used in the SMIF provided a sheet resistance value.  This value was calculated by the instrument by providing a small current $(I)$ through the 4 probes which are equally spaced on the sample and measured the voltage $(V)$ across the two inner probes.  The $V/I$ values reported by the instrument are in units of Ohms $(\Omega)$.  These values are called the sample's sheet resistance.  For thick semiconductor samples like the ones measured in the SMIF---where the thickness of the wafer $(W)$ is much smaller than the width and length of the sample---the resistivity of the sample is given by the equation

\begin{align}
\rho = \frac{V}{I} t
\end{align}

where $V/I$ is the sample's sheet resistance $(\Omega)$ and $t$ is the sample's thickness (cm).

Calculate the resistivity of each of the wafers that you measured in the SMIF using the above equation.  Enter these values along with the sheet resistance and sample thickness for each of the wafers that you measured in the SMIF.  Enter your values by clicking \href{https://duke.qualtrics.com/jfe/form/SV_bkjCbShas8s2N13}{here} or going to the url \url{https://duke.qualtrics.com/jfe/form/SV_bkjCbShas8s2N13}.

\newpage
\section{Objectives of this Laboratory}
The objectives of this laboratory session are as follows:
\begin{itemize}
\item Measure the sheet resistance of several Silicon wafers using a 4-point probe Van der Pauw measurement tool
\item Using wafer thickness, determine the resistivity of Silicon wafers
\item Determine Carrier Concentration of Silicon Wafers
\item Determine $\mu$ of Silicon Wafers
\item Obtain class data and average class values
\item Determine number and values of doped wafers used---determine $\sigma$ of each
\end{itemize}
~\\ Below, we have included a table with all of the symbols used in this lab manual and their definitions for your convenience.
\def\arraystretch{1.5}
\begin{table}[ht!]
\caption{Relevant notation used throughout lab manual}
\centering
\begin{tabular}{|c|l|l|} \hline
Symbol 	& Definition							& Unit		\\ \hline
$R$ 	& Resistance 							& $\Omega$	\\ \hline
$L$		& Length								& cm		\\ \hline
$A$		& Area									& cm$^2$	\\ \hline
$t$		& Thickness								& cm		\\ \hline
$I$		& Current								& A			\\ \hline
$V$		& Voltage								& V			\\ \hline	
$\rho$ 	& Sample resistivity 					& $\Omega \mbox{-cm}$ \\ \hline
$e$ 	& Magnitude of the electronic charge 	& C 		\\ \hline
$\mu_p$ & Hole mobility 						& cm$^2 / V_s$ \\ \hline
$\mu_n$ & Electron mobility 					& cm$^2 / V_s$ \\ \hline
$n,p$ 	& Carrier concentration 				& cm$^{-3}$	\\ \hline
$N_A$	& Acceptor Concentration 				& cm$^{-3}$	\\ \hline
$N_D$ 	& Donor Concentration 					& cm$^{-3}$	\\ \hline
$N_I$	& Impurity Concentration 				& cm$^{-3}$	\\ \hline
\end{tabular}
\end{table}

\newpage
\section{Determining Carrier Concentration for Resistivity Data}

All of the samples measured in the SMIF were p-type Boron doped.  To determine the carrier concentration, $N_A$, a few assumptions need to be made.  

Assumptions:
\begin{enumerate}
\item $\mu_p = 480 \mbox{ cm}^2/V_s$ (see Table 5.1, pp. 158 in Neaman text)
\item $N_D=0$
\item Complete ionization 
\end{enumerate}

Along with these assumptions, we can use
\begin{align}
\rho = \frac{1}{e(\mu_nn + \mu_pp)}
\end{align}

to determine carrier concentration. Here, $e$ is the magnitude of the electronic charge, which is always $1.6 \times 10^{-19}$ C.

Given the above assumptions, the equation for $\rho$ reduces to
\begin{align}
\rho = \frac{1}{e \mu_p N_A}.
\end{align}

\begin{itemize}
\item[$\square$] Calculate the carrier concentration, $N_A$, for the entire class data set. Since we are assuming that $N_D = 0$, the $N_A$ value is exactly the impurity concentration, $N_I$. 
\end{itemize}

\section{Plot of Resistivity vs. Impurity Concentration}

Plots of resistivity, $\rho$ versus impurity concentration, $N_I$ are called Irwin curves.  They are used to estimate the carrier concentrations of samples once the sample's resistivity is known.

\begin{itemize}
\item[$\square$] Plot on a log-log scale the resistivity, $\rho$ versus impurity concentration, $N_I$ (in this case, the same as the carrier concentration, $N_A$) for the class data set.  Your plot should resemble Figure 5.4 on pp. 165 of the Neamen text.  Remember that the SMIF samples were all p-type (Boron) doped.  This plot should have many values of $\rho$ and $N_I$ based on the class data.
\item[$\square$] Determine a curve-fit to this data from the above plot including $\pm$ error ranges.  Justify the $\pm$ error ranges used.  Does the expected error range vary with carrier concentration?
\end{itemize}

\newpage
\section{Estimating expected Resistivity given a Carrier Concentration}

With a plot of resistivity, $\rho$, versus impurity concentration, $N_I$, it should be possible to extract either $\rho$ or $N_I$ given the other.

\begin{itemize}
\item[$\square$] For a Si p-type (Boron) doped Carrier Concentration, $N_A$ of $1 \times 10^{17} \mbox{ cm}^3$, what range of Resistivity, $\rho$, values would you expect to obtain from this sample? Use your class averaged data plot and curve-fit from Section 3 to answer this question. Note that a range of resistivity is specified, so the $\pm$ error range must be used.
\end{itemize}

\section{Device Performance based on Doping and Sample Data}

The resistance, $R$, of a sample can be determined based on its resistivity, $\rho$, given the sample cross-sectional area, $A$, and length, $L$. The equation for determining Resistance from resistivity is the following (see Neaman Section 5.1.3 pp. 164):
\begin{align}
R = \frac{\rho L}{A}
\end{align}

\begin{itemize}
\item[$\square$] For a die-sized portion of a wafer just like the ones measured in class, with cross-sectional area $A = 10^{-6} \mbox{ cm}^2$ and length $L = 0.001$ cm of the Si p-type (Boron) doped sample above---with Carrier Concentration $N_A$ of $1 \times 10^{15} \mbox{cm}^3$ and a resistivity range as determined in Section 4 based on class-measured data---in an applied voltage of 5 V, determine the range of expected currents that would flow through this die.
\end{itemize}

~\\Submit the answers to the questions marked by a square $(\square)$ along with all of the other below-listed items in the rubric in your Lab X write-up. 

\newpage
\phantomsection
\section*{Grading Rubric}
\addcontentsline{toc}{section}{Grading Rubric}
\markboth{Grading Rubric}{Grading Rubric}
\vfill
\begin{table}[ht!]
\caption{ECE 230L Laboratory X Grading Rubric}
\centering
\begin{tabular}{l|c} \hline
Criteria & Points Possible \\ \hline \hline
\textbf{PreLab}	& \textbf{15}\\ 
Sheet resistances (x 3) reported & 5 \\ 
Sample thicknesses (x 3) reported & 5 \\ 
Resistivities calculated (x 3) & 5 \\ \hline
\textbf{Determining Carrier Concentration for Resistivity Data} & \textbf{10} \\ 
Data set & 10 \\ \hline
\textbf{Plot of Resistivity vs. Impurity Concentration (Irwin curve)} & \textbf{45} \\ 
            Plot & 10 \\ 
            Curve-fit & 10 \\ 
            $\pm$ error range specified & 10\\ 
              Justification for error range	& 15 \\ \hline
\textbf{Estimating expected Resistivity given Carrier Concentration} & \textbf{15} \\ 
Range of resistivities specified & 15 \\ \hline
\textbf{Device Performance based on Doping and Sample Data} & \textbf{15} \\ 
              Current range specified from data & 15 \\ \hline \hline
Total	& 100 \\ \hline
\end{tabular}
\end{table}
\vfill
%____________________________________________________________________________
%
%	End{PROBLEMS}
%____________________________________________________________________________
\end{document}
