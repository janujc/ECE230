\documentclass[12pt]{../manual}
%____________________________________________________________________________
%
%	TITLE AND TABLE OF CONTENTS
%____________________________________________________________________________
\begin{document}
\makeheader{Lab 3}{Fall 2019} % Change Semester and Year as needed
\begin{center}
\textbf{\huge ECE 230L - LAB 3}\\~\\
\textbf{\large INTRODUCTION TO CIRCUIT SIMULATION USING PSPICE}\\~\\
\rule{6.5in}{0.5mm}\\
\end{center}

\tableofcontents

\listoffigures

\newpage
%____________________________________________________________________________
%
%	BODY
%____________________________________________________________________________
\section{Objectives of this Laboratory}
The objectives of this laboratory session are to introduce you to the basics of PSpice by learning:
\begin{itemize}
\item How to set-up your PSpice simulation environment,
\item How to represent the circuit elements,
\item How to construct the circuits, and
\item How to simulate the circuits.
\end{itemize}
%____________________________________________________________________________
%
% 	ORCAD
%____________________________________________________________________________
\section{Setting Up a Circuit Using ORCAD Capture}
To create a circuit in a PSpice environment, one must first launch ORCAD:
\begin{enumerate}
	\item Open ORCAD Capture CIS
	\item Create a new project by selecting {\bf File} $\to$ {\bf New} $\to$ {\bf Project}
	\item Name your project `Lab 3'
	\item Choose {\bf Analog or Mixed A/D} under the {\bf Create a New Project Using} menu
	\item Select {\bf Create a blank project} when prompted
\end{enumerate}

\begin{figure}[ht!]
	\begin{center}
		\includegraphics[width=0.9\textwidth]{./figures/BlankSchematic.PNG}
	\end{center}
	\caption{Blank Schematic}
	\label{fig:blankSchematic}
\end{figure}

Once the new project has been created, circuit design can begin. Sources, components, ground nodes, and wires can be selected using the {\bf Place} menu.

PSpice will be used to model and perform analyses on the circuit in Figure \ref{fig:initCircuit}.
\begin{figure}[ht!]
	\begin{center}
		\begin{circuitikz}[scale=1.5]
			\draw 	
			(0,0) 	to[short] 										++(2,0)
					node[ground] {}
					to[short] 										++(2,0)
					to[C, l_=${C = \SI{1}{\nano\farad}}$]			++(0,2)
					to[short]		++(-2,0)
					to[R, l_=${R_1 = \SI{1}{\kilo\ohm}}$]			++(-2,0)
					to[battery, l_=$V_1$]		++(0,-2)
			(2,0)	to[R, *-*, l_=${R_2 = \SI{1}{\kilo\ohm}}$]		++(0,2)
		;\end{circuitikz}
	\end{center}
	\caption{Initial Circuit}
	\label{fig:initCircuit}
\end{figure}

To create the circuit, 
\begin{enumerate}
	\item Add a DC Voltage Source: Place $\to$ PSPice Component $\to$ Source $\to$ Voltage Source $\to$ DC
	\item Insert Resistors and Capacitors: Place $\to$ PSpice Component $\to$ Passives
	\item Insert Wires and connect circuit nodes: Place $\to$ Wire
	\item Insert a Ground: Place $\to$ Ground $\to$ 0/SOURCE
\end{enumerate}

Note that you can use Ctrl-R to rotate the components. Also, to change values of any component, double click on the component and change the values using the pop-up menu. You may also change values by double clicking directly on the value and typing a new value.
%____________________________________________________________________________
%
%	DC Analysis
%____________________________________________________________________________
\newpage
\section{DC Analysis in PSpice}
\begin{figure}[ht!]
	\begin{center}
		\includegraphics[width=0.6\textwidth]{figures/DCAnalysisCircuitCrop.PNG}
	\end{center}
	\caption{Circuit for DC Analysis}
	\label{fig:dc}
\end{figure}

To perform a DC analysis of the circuit, create a new simulation profile by selecting \textbf{PSpice} $\to$ \textbf{New Simulation Profile}. Name the new profile `dc' and press \textbf{Create}. Then,  select \textbf{DC Sweep} in the \textbf{Analysis Type} drop down menu and use the following parameters:
\begin{itemize}
\item Sweep variable $\to$ Voltage source: V1
\item Sweep Type: Linear
\item Start Value: 0
\item End Value: 10
\item Increment: 0.01
\end{itemize}

Press \textbf{Apply} and \textbf{OK} to save the profile settings. 

%Begin the simulation by selecting \textbf{PSpice} $\to$ \textbf{Run}. To view the circuit behavior at a particular point, follow \textbf{Trace} $\to$ \textbf{Add Trace} to select different values to plot. Plot V(R1:1), V(R1:2), I(R1), and I(R2). Figure \ref{fig:dcAnalRes} shows the result of DC analysis.

\begin{figure}[ht!]
\begin{center}
\includegraphics[width=\textwidth]{figures/ResultDCAnalysisCrop.PNG}
\caption[Result of DC Analysis]{Result of DC Analysis. Top: Current; Bottom: Voltage}
\label{fig:dcAnalRes}
\end{center}
\end{figure}
%____________________________________________________________________________
%
%	AC Analysis
%____________________________________________________________________________
\newpage
\section{AC Analysis in PSpice}
Before performing an AC analysis, a new AC voltage source has to replace the DC source. To change the source, delete the DC source and follow \textbf{Place} $\to$ \textbf{PSpice Component} $\to$ \textbf{Source} $\to$ \textbf{AC}. The following parameters will be used:
\begin{itemize}
\item DC Value: 10
\item AC Amplitude: 1
\end{itemize}

\begin{figure}[ht!]
	\begin{center}
		\includegraphics[width=0.6\textwidth]{figures/ACAnalysisCircuitCrop.PNG}
	\end{center}
	\caption{Circuit for AC Analysis}
	\label{fig:ac}
\end{figure}

After the voltage source properties have been changed, AC analysis can be performed. First, create a new simulation profile called `ac'. Next, select \textbf{AC Sweep/Noise} in the \textbf{Analysis Type} drop down menu and use the following parameters:
\begin{itemize}
\item AC Sweep Type: Logarithmic 
\item Select \textbf{Decade} from drop down menu below
\item Start Frequency: 1k
\item Stop Frequency: 1Meg
\item Number of Points per Decade: 10
\end{itemize}
Press \textbf{Apply} and \textbf{OK} to save the profile settings. They should look like Figure \ref{fig:acSettings}.

\begin{figure}[ht!]
	\begin{center}
		\includegraphics[width=0.6\textwidth]{figures/ACAnalysisSettings.PNG}
	\end{center}
	\caption{Settings for AC Analysis}
	\label{fig:acSettings}
\end{figure}

Now, begin the simulation by selecting \textbf{PSpice} $\to$ \textbf{Run}. % To view the circuit behavior at a particular point, follow \textbf{Trace} $\to$ \textbf{Add Trace} to select different values to plot. Plot V(R1:1), V(R1:2), I(R1), and I(R2). Figure \ref{fig:acAnalRes} shows the result of AC analysis.

\subsection*{Trace Expressions in PSpice}
Trace expressions can be used to plot the phase of a desired value, a parameter in units of dB, or plot other useful mathematical operations on circuit parameters. Trace expressions are available under the \textbf{Trace} $\to$ \textbf{Add Trace} menu. Select \textbf{Plot} $\to$ \textbf{Add Plot to Window} and plot the value of V(R1:2) in dB. Use the trace expression DB(V(R1:2)). Next, add another plot to the window and plot the phase of V(R1:2). Use the trace expression P(V(R1:2)). The resulting plots are shown in Figure \ref{fig:acAnalRes}.

\begin{figure}[ht!]
\begin{center}
\includegraphics[width=\textwidth]{figures/ResultACAnalysisCrop.PNG}
\caption{Result of AC Analysis}
\label{fig:acAnalRes}
\end{center}
\end{figure}
%____________________________________________________________________________
%
%	Transient Analysis
%____________________________________________________________________________
\newpage
\section{Transient Analysis in PSpice}
Before performing a transient analysis, replace the AC source with a Pulse source (\textbf{Place} $\to$ \textbf{PSpice Component} $\to$ \textbf{Source} $\to$ \textbf{Pulse}). The following parameters will be used to set up the pulse: 
\begin{itemize}
\item V1 = 0
\item V2 = 5 
\item TD = 10n 
\item TR = 20n 
\item TF = 20n 
\item PW = 500n 
\item PER = 2u
\end{itemize}

We include the circuit in Figure \ref{fig:trans}.

\begin{figure}[ht!]
	\begin{center}
		\includegraphics[width=0.7\textwidth]{figures/TransientAnalysisCircuitCrop.PNG}
	\end{center}
	\caption{Circuit for Transient Analysis}
	\label{fig:trans}
\end{figure}

After creating the circuit, create a new simulation profile called ``Transient''. To analyze the new circuit, select \textbf{Time Domain (Transient)} in the \textbf{Analysis Type} drop down menu and use the following parameters:
\begin{itemize}
\item Run to Time: 2u
\item Start saving data after: 0
\item Maximum step size: 10n
\end{itemize}
Press \textbf{Apply} and \textbf{OK} to save the profile settings. 

\begin{figure}[ht!]
	\begin{center}
		\includegraphics[width=0.6\textwidth]{figures/TransientAnalysisSettings.PNG}
	\end{center}
	\caption{Settings for Transient Analysis}
	\label{fig:transSettings}
\end{figure}

Begin the simulation by selecting \textbf{PSpice} $\to$ \textbf{Run}. Plot the source voltage, V(R1:1), and V(R1:2).

\begin{figure}[ht!]
	\begin{center}
		\includegraphics[width=\textwidth]{figures/ResultTransientAnalysisCrop.PNG}
	\end{center}
	\caption{Result of Transient Analysis}
	\label{fig:transAnalRes}
\end{figure}
%____________________________________________________________________________
%
%	Practice Example
%____________________________________________________________________________
\newpage
\section{Practice Example}
Use PSpice to simulate the circuit shown in Figure \ref{fig:practice} using DC, AC, and Transient analyses. Take a screencap of a plot of the voltage across capacitor $C_1$ in each analysis.

\begin{figure}[ht!]
\begin{center}
\begin{circuitikz}[scale=1.5]
%\ctikzset{resistors/scale=1.5,batteries/scale=1.5,grounds/scale=2.0,diodes/scale=1.5}
\draw
(0,0) 	to[short, *-] 	++(0,1)
		to[C, l=${C_1=\SI{1}{\nano\farad}}$]	++(2,0)
		to[short, -*]	++(0,-1)
		to[short]		++(1,0)
		to[R, l=${R_{sec}=\SI{1}{\kilo\ohm}}$] 	++(0,-2)
		to[short, -*]		++(-2,0)
		node[ground] {}
		to[short]		++(-2,0)
(2,0)	to[R, l=${R_{fir}=\SI{1}{\kilo\ohm}}$] 	++(-2,0)
		to [short] ++ (-1,0)
		to[battery, l_=${v_1}$, i=${i_{IN}}$] 	++(0,-2)
;\end{circuitikz}
\caption{Practice Circuit}
\label{fig:practice}
\end{center}
\end{figure}

\begin{enumerate}
\item In the DC analysis, use the following parameters:
\begin{itemize}
\item Sweep Type: Linear
\item Start Value: 0
\item End Value: 10
\item Increment: 0.01
\end{itemize}
\item In the AC analysis, change the voltage source to have the following values:
\begin{itemize}
\item DC Value: 5
\item AC Amplitude: 1
\end{itemize}
And use the following parameters:
\begin{itemize}
\item AC Sweep Type: Logarithmic 
\item Start Frequency: 10
\item Stop Frequency: 1Meg
\item Number of Points per Decade: 10
\end{itemize}
\item In the transient analysis, change the voltage source to have the following values:
\begin{itemize}
\item V1 = 0, 
\item V2 = 5, 
\item TD = 100n, 
\item TR = 40n, 
\item TF = 40n, 
\item PW = 500n, 
\item PER = 2u
\end{itemize}
And use the following parameters:
\begin{itemize}
\item Run to Time: 2u
\item Start saving data after: 50n
\item Maximum step size: 10n
\end{itemize}
\end{enumerate}
%____________________________________________________________________________
%
%	Exploration
%____________________________________________________________________________
\newpage
\section{Exploration: Th\'evenin Equivalent Circuits}
\subsection*{Purpose}
The purpose of this exercise is to learn how to form a Th\'evenin Equivalent circuit by using circuit parameters obtained during simulation.

\subsection*{Introduction}
Any linear DC circuit as seen at a pair of terminals can be reduced to a practical voltage source (an ideal voltage source in series with a resistor).

\begin{figure}[ht!]
\begin{center}
\begin{circuitikz}[scale=1.5]
%\ctikzset{resistors/scale=1.5,batteries/scale=1.5,grounds/scale=2.0,diodes/scale=1.5}
\draw
(0,0)	to[R, *-, l=${R_{TH}}$]		++(-4,0)
		to[battery, l=${V_{OC}}$] 	++(0,-2)
		to[short, -*]	++(4,0)
(0,0.25)	node[] {A}
(0,-1.75)	node[] {B}
;\end{circuitikz}
\caption{Th\'evenin Equivalent Circuit}
\label{fig:thev}
\end{center}
\end{figure}

To form a Th\'evenin Equivalent circuit, two quantities must be calculated, measured, or simulated:
\begin{itemize}
\item $v_{oc}$: The open circuit voltage drop from terminals A to B
\item $i_{sc}$: The short circuit current from terminals A to B
\end{itemize}
Once the values for $v_{oc}$ and $i_{sc}$ have been obtained, the Th\'evenin resistance $R_{TH}$ can be determined using the relation:
\begin{align}
R_{TH} = \frac{v_{oc}}{i_{sc}} \label{eq:rth}
\end{align}
If the circuit contains no dependent sources, then $R_{TH}$ may also be found by turning off all of the independent sources and using resistance reduction at terminals A and B.

\newpage
\subsection{Example Exercise}
Simulate the circuit in Figure \ref{fig:exampleCircuit} and form its Th\'evenin Equivalent as seen from terminals A and B.

\begin{figure}[ht!]
\begin{center}
\includegraphics[width=0.7\textwidth]{figures/ExampleCircuitSchematicCrop.PNG}
\caption{Example Circuit}
\label{fig:exampleCircuit}
\end{center}
\end{figure}

To find the open circuit voltage, replace resistor $R_4$ with a large ``dummy'' resistance (at least \SI{100}{\mega\ohm}). Create a new simulation profile called ``thev''. Choose \textbf{Bias Point} under the \textbf{Analysis type} menu and check the box labeled \textbf{Include Detailed Bias Point Information} under the \textbf{Output File Options} heading. Press \textbf{Apply} and \textbf{OK} to save the profile. Now, run the simulation. Once the simulation is complete, follow \textbf{PSpice} $\to$ \textbf{Bias Points} and click \textbf{Enable}. Record the voltage across terminals A and B as your open circuit voltage.

\begin{figure}[ht!]
\begin{center}
\includegraphics[width=0.7\textwidth]{figures/OpenCircuitSchematicCrop.PNG}
\caption{Open Circuit Voltage Schematic}
\label{fig:openCircuit}
\end{center}
\end{figure}

To find the short circuit current, remove $R_4$ and connect terminals A and B with a wire.
Use the same ``thev'' simulation profile that was created to find $v_{oc}$. Run the simulation. Once the simulation is complete, enable bias points. This will show the node voltages and currents throughout the circuit. Record the current between terminals A and B as your short circuit current, $i_{sc}$.

\begin{figure}[ht!]
\begin{center}
\includegraphics[width=0.7\textwidth]{figures/ShortCircuitSchematicCrop.PNG}
\caption{Short Circuit Voltage Schematic}
\label{fig:shortCircuit}
\end{center}
\end{figure}

Calculate the Th\'evenin resistance using Equation \ref{eq:rth}. Now, re-create Figure \ref{fig:thev} in PSpice using your $v_{oc}$ and $R_{TH}$ values. Place $R_4$ back into your first circuit and place a resistor of equal value between terminals A and B in your Th\'evenin Equivalent. Run the simulation using the same ``thev'' profile. Once the simulation is complete, enable bias points. Check to make sure the terminal voltages and currents match for both circuits.

\begin{figure}[ht!]
\begin{center}
\includegraphics[width=0.5\textwidth]{figures/TheveninEquivalentExampleCrop.PNG}
\caption{Th\'evenin Equivalent Schematic}
\label{fig:thevCircuit}
\end{center}
\end{figure}

\newpage
\subsection{Challenge Exercise: Th\'evenin Equivalent Circuit}
Create the schematic in Figure \ref{fig:exerciseCircuit} using PSpice and find its Th\'evenin Equivalent circuit as seen from nodes A and B. In your lab report, be sure to include all voltages and currents present in the Exercise Circuit and its Th\'evenin Equivalent, as well as a short explanation of each step you took in finding the Th\'evenin Equivalent.
\begin{figure}[ht!]
\begin{center}
\begin{circuitikz}
%\ctikzset{grounds/scale=1.5}
\draw
(0,0) 	to[short] 		++(4,0)
		node[ground] {}
		to[short] 		++(4,0)
		to[R, l=$R_5$, a=\SI{500}{\ohm}]	++(0,4)
		to[R, l=$R_3$, a=\SI{200}{\ohm}]	++(0,4)
		to[short]		++(-8,0)
		to[battery, a=$V_1$, l=\SI{1}{\volt}]	++(0,-8)
(4,0)	to[R, *-, l=$R_2$, a=\SI{1}{\kilo\ohm}]			++(0,4)
		to[R, -*, l=$R_1$, a=\SI{100}{\ohm}]			++(0,4)
(4,4)	to[R, *-*, l=$R_4$, a=\SI{300}{\ohm}]			++(4,0)
(3.5,4) node[] {\textcolor{red}{A}}
(8.5,4) node[] {\textcolor{red}{B}}
;\end{circuitikz}
\caption{Exercise Circuit}
\label{fig:exerciseCircuit}
\end{center}
\end{figure}
%____________________________________________________________________________
%
%	Grading Rubric
%____________________________________________________________________________
\newpage
\phantomsection
\addcontentsline{toc}{section}{Grading Rubric}
\markboth{Grading Rubric}{Grading Rubric}
\hspace{0pt}
\vfill % used to center table vertically on page
\begin{table}[ht!]
\caption{ECE 230L Laboratory 3 Grading Rubric}
\centering
\begin{tabular}{l|c} \hline
Criteria & Points Possible \\ \hline \hline
\textbf{DC Analysis}			& \textbf{10} \\
Circuit Diagram 				& 5 \\
Waveforms 						& 5 \\ \hline
\textbf{AC Analysis}			& \textbf{10} \\
Circuit Diagram 				& 5 \\
Waveforms 						& 5 \\ \hline
\textbf{Transient Analysis}		& \textbf{10} \\
Circuit Diagram 				& 5 \\ 
Waveforms 						& 5 \\ \hline
\textbf{Practice Exercise}		& \textbf{35} \\
Circuit Diagram 				& 5 \\
DC Analysis						& 10 \\
AC Analysis						& 10 \\
Transient Analysis				& 10 \\ \hline
\textbf{Th\'evenin Equivalent Example Circuit} & \textbf{20} \\
Circuit Diagram					& 10 \\
$V_{OC}$ and $I_{SC}$ Labeled	& 5 \\
Correct $R_{TH}$ Value			& 5 \\ \hline
\textbf{Th\'evenin Equivalent Challenge Circuit} & \textbf{15} \\
Circuit Diagram					& 5 \\
$V_{OC}$ and $I_{SC}$ Labeled	& 5 \\
Correct $R_{TH}$ Value			& 5 \\ \hline \hline
Total							& 100 \\ \hline
\end{tabular}
\end{table}
\vfill % used to center table vertically on page
\end{document}
