\documentclass[12pt]{../manual}
%____________________________________________________________________________
%
%	TITLE AND TABLE OF CONTENTS
%____________________________________________________________________________
\begin{document}
\makeheader{Lab 3}
\begin{center}
\textbf{\huge ECE 230L - LAB 3}\\~\\
\textbf{\large ELECTRICAL CHARACTERIZATION AND PARAMETER EXTRACTION OF SILICON PN-JUNCTION DIODES}\\~\\
\rule{6.5in}{0.5mm}\\
\end{center}

\tableofcontents

\listoffigures

\newpage
%____________________________________________________________________________
%
%	BODY
%____________________________________________________________________________
\section{Objectives of this Laboratory}
The objectives of this laboratory session are as follows:
\begin{itemize}
\item To measure the $I_D(V_{PN})$ current-voltage characteristics of a PN-junction diode using standard test and measurement equipment,
\item To understand the measurement of the $I_D(V_{PN})$ current-voltage characteristics of a PN-junction diode using the LabVIEW platform,
\item To evaluate the limitations of the electrical measurements as compared to theory,
\item To explore one technological application of PN-junction diodes in the form of power electronics (half-wave and bridge rectifiers)
\end{itemize}
\textit{Note: A detailed introduction to LabVIEW can be found on Sakai.}

\section{Electrical Characterization of Silicon PN-Junction Diode}
In this laboratory, you will measure the static current-voltage $I_D(V_{PN})$ characteristics of silicon PN-junction diode 1N4148 manually with test and measurement equipment in the lab kit. The measurement set-up includes the following:
\begin{itemize}
\item JAMECO JE25 Solderless Breadboard
\item WISHER Wire Kit (various sizes and colors)
\item UCTRONICS U6225 Power Supply Module
\item AstroAI AM33D Digital Multimeter
\end{itemize}
A cross-section and circuit symbol of the PN-junction diode is shown in Figure \ref{fig:diode}.
\begin{figure}[ht!]
\centering
\begin{tabular}{c}
\begin{tikzpicture}[scale=2,european]
\ctikzset{resistors/scale=3, resistors/thickness=6}
\draw (0,0) node[anchor=south] {Anode}
			to[R, o-o] ++(4.5,0)
			to node[anchor=south] {Cathode} ++(0,0);
\fill[black] (2.25,0.33)	rectangle (2.30, -0.33);
\draw 	(1.85,0)	node[anchor=center] {$P$}
		(2.70,0)	node[anchor=center] {$N$}
;\end{tikzpicture} \\
\vspace{5mm}\\
\begin{circuitikz}[scale=2]
\ctikzset{diodes/scale=2, grounds/scale=2}
\draw 
(0,0) 	node[anchor=south] {Anode}
		to [D*, o-o] ++(4.5,0)
		to node[anchor=south] {Cathode} ++(0,0)
;\end{circuitikz}
\end{tabular}
\caption[PN-Junction Diode Schematics]{PN-Junction Diode Cross-Section and Symbol}
\label{fig:diode}
\end{figure}

The arrow of the diode indicates the direction of the current.

Using the multimeter, obtain the $I_D(V_{PN})$ static current and voltage characteristics of the silicon PN-junction diode 1N4148 by carrying out the following steps:
\begin{enumerate}
\item Obtain a 1N4148 diode.
\item The minimum output voltage by the Power Supply Module is 0.6V. To provide a low enough voltage to see the diode turn on, a voltage divider circuit is needed. Build the circuit shown in Figure \ref{fig:diodeTest} using the UCTRONICS U6225 Power Supply Module and the AstroAI AM33D Digital Multimeter as an Ammeter (measure with the 2000$\mu$A setting)  with R1 = 680$\Omega$, R2 = 100$\Omega$, and R3 = 1000$\Omega$. 

\begin{itemize}
\item[$\square$] What fraction of the source voltage $V_{S}$ goes across R3 and the diode?
\end{itemize}

\begin{figure}[ht!]
\centering
\begin{circuitikz}[scale=2]
\ctikzset{resistors/scale=1.5,batteries/scale=1.5,instruments/scale=1.5,diodes/scale=1.5}
\draw
(0,4) 	to[battery, l_=$V_{S}$, i=$i_S$] ++(0,-4)
(0,4) -- (2,4)
(2,4) 	to[R, l^=$R_1$] (2,2)
(2,2) 	to[R, l_=$R_3$] (4,2)
(4,2)	to[rmeter, t=A] (4,0)
(4,0)	to[D*, l_=$D$]	(2,0)
(2,0)	to[R, l_=$R_2$, v^<=$V_{DC}$] (2,2)
(2,0) -- (0,0)
;\end{circuitikz}
\caption{PN-Junction Diode Test Circuit}
\label{fig:diodeTest}
\end{figure}


\item Start with 1.00V on the Power Supply Module. Increase the voltage to the circuit slowly and record the current through the diode. Record to 8V on the Power Supply Module.
\begin{itemize}
\item[$\square$]Determine approximately the voltage at which the diode begins to conduct current. This is known as the knee voltage.
\end{itemize}
\item Create a plot of measured $V_{DC}$ vs. $I$ for this circuit. Remember that the $V_{DC}$ experienced by R3 and the diode is not the voltage displayed on the Power Supply Module.

\item To account for the accuracy of your voltage source in producing its values as set by the Power Supply Module, it is important to measure the voltage $V_{PN}$ directly across the diode. Doing so will give you the actual voltage across the diode for each input voltage step from the Power Supply Module. This voltage, $V_{PN}$, will be plotted versus $I_D$ to obtain the diode I-V curve later. Repeat steps 3 and 4 but record the voltage $V_{PN}$ across the diode and plot $V_{DC}$ vs. $V_{PN}$ instead. Your circuit should look like Figure 3 below. Be sure to use the same start and stop values as well as step size as before.

\begin{figure}[ht!]
\centering
\begin{circuitikz}[scale=2]
\ctikzset{resistors/scale=1.5,batteries/scale=1.5,instruments/scale=1.5,diodes/scale=1.5}
\draw
(0,4) 	to[battery, l_=$V_{S}$, i=$i_S$] ++(0,-4)
(0,4) -- (2,4)
(2,4) 	to[R, l^=$R_1$] (2,2)
(2,2) 	to[R, l_=$R_3$] (4,2)
(4,2)	to[D*, *-*, l_=$D$] (4,0)
(4,0) -- (0,0)
(2,0)	to[R, l_=$R_2$, v^<=$V_{DC}$] (2,2)
(4.25,0) node[currarrow, rotate=180]{}
		to [short] ++(0.75,0)
		to [rmeter, t=V] ++(0,2)
		to [short] ++(-0.75,0) node[currarrow, rotate=180]{}
;\end{circuitikz}
\caption{PN-Junction Diode Test Circuit with Voltmeter across Diode}
\end{figure}

\item Due to the limitations of our equipment, we will not be repeating the above experiment to measure the reverse-bias current-voltage characteristsics of the diode. However, one way to do this would be to reverse the polarity of the power supply in the above circuits.

\item Normally, the current and voltage characteristics would be obtained using LabVIEW, which can more accurately measure the above diode characteristics. LabVIEW can automatically limit the maximum value of the current that passes through the diode. By using the general-purpose interface bus (GPIB) cable connecting the computer with the measurment equipment, LavVIEW can control digital instruments such as an Agilent E3631A Triple Output DC Power Supply and Agilent 34410A Multimeter.

\end{enumerate}

\newpage
\section{Theoretical Curve of CV Characteristics of PN-Junction Diode}
Circuit simulation is an important tool in the analysis and design of microelectronic circuits. Spice is a general-purpose circuit simulation program in which nonlinear dc, nonlinear transient, and linear ac analyses of electronic circuits are carried out. Circuits may contain resistors, capacitors, inductors, mutual inductors, independent voltage and current sources, dependent sources, lossless and lossy transmission lines, switches, diodes, BJTs, JFETs, MESFETs, and MOSFETs. The version of Spice used in the ECE Department at Duke is PSpice. You will be using PSpice over the next several labs, starting next week. Below, we include Table \ref{table:params} as an introduction to PSpice and the model parameters for a typical PN-Junction Diode in PSpice. 

\begin{table}[ht!]
\caption{Typical PN-Junction Model Parameters}
\centering
\begin{tabular}{|c|c|c|c|} \hline
Parameter & Description & Nominal Value & Unit \\ \hline \hline
$I_S$ & Diffusion Saturation Current & $1.0 \times 10^{-14}$ & A \\ \hline
$\gamma$ & Ideality Factor & 1.0 & --- \\ \hline
$R_S$ & Series Resistance & 0 & $\Omega$ \\ \hline
$E_g$ & Band Gap Energy & 1.17 & eV \\ \hline
$V_B$ & Reverse Breakdown Voltage & $\infty$ & V \\ \hline
$I_D(V_B)$ & Current at Breakdown Voltage & $1.0 \times 10^{-3}$ & A \\ \hline
$\tau_0$ & Minority-Carrier Lifetime & 0 & S \\ \hline
$C_j$ & Zero-Bias Depletion Capacitance & 0 & F \\ \hline
\end{tabular}
\label{table:params}
\end{table}

There are a large number of parameters used in the representation of a PN-junction
diode, but only the most commonly used parameters will be considered here. The first
two parameters are related to the representation of the diode current-voltage behavior
assumed to be in the following form:
\begin{align}
I_D(V_{PN}) = I_S \exp\left(\frac{q V_{PN}}{\gamma k_b T} - 1\right) \label{eq:ID}
\end{align}
This relationship is written to obtain an empirical fit to experimental data and was not derived from physics-based principles. For $V_{PN} > 3\frac{k_BT}{q}$, which is
approximately 75 mV at room temperature, Equation (\ref{eq:ID}) reduces to
\begin{align}
I_D(V_{PN}) \approx I_S \exp\left(\frac{q V_{PN}}{\gamma k_b T}\right). \label{eq:redID}
\end{align}
The values most commonly specified include the following: the saturation current, $I_S$, the emission coefficient, $\gamma$, the series resistance $R_S$, the junction capacitance $C_j$, the transmit time, $\tau_0$ (default = 0 sec), the reverse-bias breakdown voltage, $V_B$, and the reverse-bias breakdown current, $I_D(V_B)$.

The ideality factor $\gamma$ in Equation \ref{eq:redID} may be found from the following formula:
\begin{align}
\gamma = \frac{V_{PN2} - V_{PN1}}{\frac{k_bT}{q}\ln\left(\frac{I_D(V_{PN2})}{I_D(V_{PN1})}\right)}
\end{align}
Note that the experimentally measured ideality factor $\gamma$ often ranges from 1 to 2.

A third parameter is the series resistance $R_S$. Another parameter is the band gap energy, sometimes called the energy gap, $E_G$, which will depend on the semiconductor being used and the temperature. We also have the reverse bias breakdown voltage, $V_B$, and the current at the breakdown voltage, $I_D(V_B)$. These first six parameters describe the static behavior of the PN-junction diode. Both $V_B$ and $I_D(V_B)$ are specified as positive numbers in PSpice, but in reality have negative values.

The final two parameters we will mention are transit time $\tau_0$ and the zero-bias depletion capacitance $C_j$. They describe the $C_{\mathrm{pn}}(V_{PN})$ characteristics of the PN-junction diode. The total capacitance $C_{\mathrm{pn}}$ of the PN-junction diode is the sum of its depletion capacitance $C_{\mathrm{pn-dep}}(V_{PN})$ and diffusion (or storage) capacitance $C_{\mathrm{pn-diff}}(V_{PN})$. If a PN-junction diode is reverse-biased, the depletion capacitance then dominates. If the PN-junction diode is forward-biased, both the depletion and diffusion capacitances are important.

% Here, we want to add a section detailing how to extract parameters from graph

\newpage
\section{Exploration}
\subsection{PN-Junction Diode Half-Wave and Bridge Rectifiers}
\subsubsection*{Half-Wave Rectifier}
One application of a PN-junction diode is to allow only the positive-going voltages of a time-varying input-voltage signal to pass to the output. Because a PN-junction diode allows current to flow through a circuit only in the forward direction, it can be used to allow the positive-going portions of an input signal to pass to the output while blocking the negative-going portions of the waveform. This kind of circuit is called a rectifier.
\begin{figure}[ht!]
\centering
\begin{circuitikz}[scale=2]
\ctikzset{resistors/scale=1.5,batteries/scale=1.5,diodes/scale=1.5}
\draw
(0,2) 	node[msport,circuitikz/RF/scale=-2](ADC1){Ch. A}
(4,2) 	node[msport,circuitikz/RF/scale=2](ADC2){Ch. B}
(0,0) 	node[msport,circuitikz/RF/scale=-2](ADC3){V}
(2,2) 	to[D*, l_=$D$, i_<=$i_{\mathrm{Ch. A}}$] (ADC1.left)
(2,2) -- (ADC2.left)
(2,2)	to[R, l_=$R$, v^=$V_O$] (2,0) -- (ADC3.left)
;\end{circuitikz}
\caption{PN-Junction Diode Half-Wave Rectifier}
\label{fig:halfRec}
\end{figure}
\begin{enumerate}
\item Build a half-wave diode rectifier circuit using a Si 1N4148 diode, as shown in Figure
4. Use a 5 V$_{\mathrm{pp}}$ sinusoidal input waveform from the M1K ADALM1000 by setting channel A to "Source Voltage, Measure Current" and producing a sine wave with parameters of 5V amplitude and 100Hz frequency. To create a waveform from -2.5V to 2.5V, a fixed 2.5V supply from the M1K will be used as demonstrated in Figure 4. Use a resistor value of $\SI{1}{\kilo\ohm}$ at the output. Connect Channel A GND to Channel B GND.
\begin{itemize}
\item[$\square$] Observe the output voltage of the circuit as measured across the resistor in Channel B on the M1K. You should see that the positive-going portions of the input-voltage waveform are allowed to pass to the output node, while the negative-going portions of the waveform are not. Attach a screenshot of the results in your report.
\item[$\square$] Compare the peak voltage from the ground reference (GND or 0 V) in Channel A screen to V$_{\mathrm{peak}}$ of the rectified waveform in Channel B. What do you notice about the maximum voltage of the rectified waveform? Is it greater than or smaller than the maximum peak voltage of the input waveform? What is the difference in Volts? Is this difference value a characteristic of the diode behavior?
\item[$\square$] Lastly, add a large capacitor (1 uF) to the output of this circuit to connect $v_o$+ to GND (i.e. in parallel with the output resistor, R). Now observe the output waveform. What has happened? How could this circuit be useful? Attach a screenshot of the results in your report.
\end{itemize}
\item Repeat the above experiment using a Ge diode.
\end{enumerate}
\newpage
\subsubsection*{Bridge Rectifier}
It is possible to improve upon the above circuit by using both halves of the sinusoidal input voltage. To do so, more than one diode is required. An intermediate improvement would be to add a second diode: this circuit is referred to generally as Full Wave Rectifier. A common and even better implementation of the Full Wave Rectifier utilizes four diodes: it is referred to as a Bridge Rectifier. (The reason it is even better than a common Full Wave Rectifier is because it does not require a centrally tapped DC reference point.) Figure \ref{fig:bridgeRec} shows a Bridge Rectifier circuit. To provide a new ground for the full-wave rectified circuit, a transformer between the power source and the diodes is required. This allows both the positive- and negative-going portions of the input-voltage waveform to pass to the output node.
\begin{figure}[ht!]
\centering
\begin{circuitikz}
\ctikzset{quadpoles/transformer core/inner=1, quadpoles/transformer core/width=1, quadpoles/transformer core/height=2.865}
\draw
(0,0)	node[transformer core, anchor=A1](T){}
(T.base) node{$1:1$}
(T.A1)	to[short] ++(-4,0)
		to[battery, l^=${v_S=V_P\sin\omega t}$] (-4,|-T.A2)
		to[short] (T.A2)
(T.B2)	to[short] ++(4.6,0)
		to[D*, *-*, l_=$D_4$] ++(2,2)
		to[short] ++(1,0)
		to[R, l^=$R_L$] ++(0,-2) node[ground] {}
(T.B1)	to[short] ++(4.6,0)
		to[D*, l^=$D_3$] ++(2,-2)
(4,-2)	to[short] ++(-1,0) node[ground] {}
(4,-2)	to[D*, -*, l^=$D_2$] ++(2,2)
(4,-2)	to[D*, *-, l_=$D_1$] ++(2,-2)	
;\end{circuitikz}
\caption{PN-Junction Diode Bridge Rectifier}
\label{fig:bridgeRec}
\end{figure}

\newpage
\section{Questions}
\begin{enumerate}
\item Define all parameters and variables in Equation (\ref{eq:redID}).
\item Plot the $I_D(V_{PN})$ characteristics obtained on a log-linear plot. Note that the term `log-linear' is used to describe a semi-log plot with a logarithmic scale on the y-axis and a linear scale on the x-axis. The y-axis label should read `1N4148 Diode Current, $I_D$ (A)' and the x-axis label should read `Diode Voltage, $V_{PN}$ (V).' Make sure to have enough data points to generate a smooth curve. Use the measured value of $V_{PN}$ (from your $V_{PN}$ vs $V$ plot) across the diode in this plot, i.e., do not assume that the voltages supplied by the power supply are those across the diode.
\item From the above log-linear plot of the I-V characteristics of the 1N4148, determine the saturation current, $I_S$, the ideality factor, $\gamma$, and the series resistance, $R_S$. (refer to the slides on Sakai)
\item To the above plot, add the theoretical PN-junction diode $I_D(V_{PN})$ characteristic curve. Use the parameters calculated in question 3. Vary the parameters, if necessary, to match the \textbf{log-linear} portion of the experimentally measured data.
\item Comment on the values of $I_S$ and $\gamma$ that yield the best fit to the data. How do your experimental and varied values compare with those given in the diode manufacturer data sheets? (For $R_s$ use Figure 2, curve 2, on the Vishay datasheet. $I_S$ can be found under reverse current, and $\gamma$ should be about 2.3.)
\end{enumerate}
\textit{Finally, be sure to add the screenshots and answers to all questions marked by $\square$ from the Exploration section.}

\newpage
\section{Extension}
To earn up to an additional 7 points (bringing the total possible points up to 100), you must show evidence of independent thought. This may include (but is not limited to):
\renewcommand\labelitemi{\textbullet}
\begin{itemize}
\item Dive deeper into your exploration. This could include researching more about the bridge rectifier and possibly building one during lab.
\item Explore the relationship in Equations (\ref{eq:ID}) and (\ref{eq:redID}). This may include finding out who first derived the equations and when it is appropriate to use them.
\item Explain how the 1N4148 diode works. How does it compare to the other diodes available in lab? When might we want to instead use a Schottky diode or a Zener diode?
\item Explore in depth another question that piqued your interest as you completed this laboratory.
\end{itemize}
The number of additional points awarded depends on the quality and extent of your independent contribution.

\newpage
\phantomsection
\addcontentsline{toc}{section}{Grading Rubric}
\markboth{Grading Rubric}{Grading Rubric}
\hspace{0pt}
\vfill % used to center table vertically on page
\begin{table}[ht!]
\caption{ECE 230L Laboratory 3 Grading Rubric}
\centering
\begin{tabular}{l|c} \hline
Criteria & Points Possible \\ \hline \hline
\textbf{Characterization of Silicon PN-Junction Diode}	& \textbf{15} \\ 
Circuit Diagram (all components labeled) 				& 1 \\ 
 Fraction of $V_{S}$ across R3 and the diode			& 2 \\ 
Knee Voltage											& 2 \\ 
Plot of $V_{DC}$ vs. $I$								& 5 \\ 
Plot of $V_{DC}$ vs. $V_{PN}$									& 5 \\ \hline
\textbf{Question 1}										& \textbf{10} \\ \hline
\textbf{Question 2}										& \textbf{10} \\ \hline
\textbf{Question 3}										& \textbf{10} \\ \hline
\textbf{Question 4}										& \textbf{10} \\ \hline
\textbf{Question 5}										& \textbf{10} \\ \hline
\textbf{Exploration}									& \textbf{28} \\ \hline
\textbf{Extension}										& \textbf{7} \\ \hline \hline
\textbf{Total}											& \textbf{100} \\ \hline
\end{tabular}
\end{table}
\vfill % used to center table vertically on page
\end{document}
